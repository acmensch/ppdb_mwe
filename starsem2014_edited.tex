%
% File twocolumn.tex
%
% Contact: v.basile@rug.nl
%%
%% Based on the style files for COLING-2014, which were, in turn,
%% Based on the style files for ACL-2014, which were, in turn,
%% Based on the style files for ACL-2013, which were, in turn,
%% Based on the style files for ACL-2012, which were, in turn,
%% based on the style files for ACL-2011, which were, in turn, 
%% based on the style files for ACL-2010, which were, in turn, 
%% based on the style files for ACL-IJCNLP-2009, which were, in turn,
%% based on the style files for EACL-2009 and IJCNLP-2008...

%% Based on the style files for EACL 2006 by 
%%e.agirre@ehu.es or Sergi.Balari@uab.es
%% and that of ACL 08 by Joakim Nivre and Noah Smith

\documentclass[11pt]{article}
\usepackage{starsem2014}
\usepackage{times}
\usepackage{url}
\usepackage{latexsym}
\usepackage{graphicx}
\usepackage{caption}
\usepackage{subcaption}
\usepackage{multibib}


%\setlength\titlebox{5cm}

% You can expand the titlebox if you need extra space
% to show all the authors. Please do not make the titlebox
% smaller than 5cm (the original size); we will check this
% in the camera-ready version and ask you to change it back.

\newcommand{\wsname}{*SEM 2014}
\newcommand{\submissionpage}{\url{http://starsem.webhosting.rug.nl/wp/?page_id=22}}
\newcommand{\filename}{starsem2014}
\newcommand{\contact}{v.basile@rug.nl}

\title{An Analysis of Multiword Expressions in the Paraphrase Database}

\author{First Author \\
  Affiliation / Address line 1 \\
  Affiliation / Address line 2 \\
  Affiliation / Address line 3 \\
  {\tt email@domain} \\\And
  Second Author \\
  Affiliation / Address line 1 \\
  Affiliation / Address line 2 \\
  Affiliation / Address line 3 \\
  {\tt email@domain} \\}

\date{}

\begin{document}
\maketitle
\begin{abstract}
We hypothesize that paraphrases may be a useful resource for understanding multiword expressions (MWEs).  We analyze the paraphrases in PPDB, the Paraphrase Database, where multiple words are re-written as a single word.  By automatically mapping from multiword expressions onto single words, NLP systems could potentially process the re-written text more easily than the original text containing MWEs. We use the MWE categorization system as described in \newcite{Sag2002} to identify paraphrases in PPDB that might be interesting to MWE researchers.  Although only a relatively small proportion of the many-to-one paraphrases in PPDB are classic MWEs, the resource contains millions of entries.  We train a classifier to distinguish the interesting MWEs from other sorts of many-to-one paraphrases. We do a pilot study on parsing paraphrased sentences, to determine whether sentences with a MWE replaced by its paraphrase are parsed correctly more often than the original.

\end{abstract}

\section{Introduction}
\label{intro}

%
% The following footnote without marker is needed for the camera-ready
% version of the paper.
% Comment out the instructions (first text) and uncomment the 8 lines
% under ``final paper" for your variant of English.
% 
\blfootnote{
    %
    % for review submission
    %
    \hspace{-0.65cm}  % space normally used by the marker
    Place licence statement here for the camera-ready version, see
    Section~\ref{licence} of the instructions for preparing a
    manuscript.
    %
    % % final paper: en-uk version (to license, a licence)
    %
    % \hspace{-0.65cm}  % space normally used by the marker
    % This work is licensed under a Creative Commons 
    % Attribution 4.0 International Licence.
    % Page numbers and proceedings footer are added by
    % the organisers.
    % Licence details:
    % \url{http://creativecommons.org/licenses/by/4.0/}
    % 
    % % final paper: en-us version (to licence, a license)
    %
    % \hspace{-0.65cm}  % space normally used by the marker
    % This work is licenced under a Creative Commons 
    % Attribution 4.0 International License.
    % Page numbers and proceedings footer are added by
    % the organizers.
    % License details:
    % \url{http://creativecommons.org/licenses/by/4.0/}
}

Multiword expressions (MWEs) are phrases whose meanings are different than the literal interpretation of the words in the phrase. MWEs include verb-particle constructions, fixed expressions, compound nominals, and decomposable idioms, to name a few \cite{Sag2002}.  MWEs are difficult to process both for non-native speakers of English, as well as for NLP systems. Studies using an eye-movement paradigm have found that non-native speakers of English required more time to retrieve figurative senses of phrases than literal ones, whereas native speakers retrieved the idiomatic meaning faster than the literal meaning \cite{siyanova-chanturia-martinez:2014}. These studies imply that L2 speakers of English may find it more difficult to understand MWEs than a similar phrase whose meaning was literal.  Among NLP systems, both parsers and information retrieval systems make errors on MWEs. \newcite{Baldwin20042} found that, among a random sample of 20,000 strings from the written portion of the British National Corpus \cite{Burnard2000}, using the English Resource Grammar \cite{Copestake+Flickinger2000}, MWEs caused 8\% of all parse errors. When manually selected compound nominals were searched for as single terms it improved information retrieval results \cite{Acosta2011}.

Because MWEs are challenging for many NLP tasks, automatically identifying them could be useful for identifying and averting errors.  Several research efforts have examined this topic.  For example, \newcite{li-sporleder:2010:NAACLHLT}  and \newcite{muzny-zettlemoyer:2013:EMNLP} built a classifiers to identify idioms.  In this paper, we use the Paraphrase Database (PPDB) as a resource to define a MWE lexicon, which could be incorporated into other NLP systems.  In addition to being a potentially useful resource for {\it identifying} MWEs, it has the unique feature of potentially giving an {\it interpretation} of the MWEs by replacing them with a one word paraphrase.  

In this paper we
\begin{itemize}
\item Analyze the PPDB for the prevalence of various types of MWEs
\item Build a classifier that distinguishes interesting MWEs in the PPDB from more generic paraphrases
\item Investigate whether parse quality can be improved by substituting MWEs with one word paraphrases
\end{itemize}


\begin{figure*}
\begin{center}
\includegraphics[width=\linewidth]{figs/paraphrase-by-pivoting}
\caption{The paraphrase database contains one-to-many paraphrase where a single English word is paraphrased as a multiword phrase.  The paraphrase of {\it imprisoned} as {\it thrown into jail} results because the shared foreign phrase {\it festgenommen} is sometimes translated using one word and sometimes as a phrase. We analyze PPDB's one-to-many paraphrases for their prevalence of more interesting MWEs.  }\label{bilingual-pivoting}
\end{center}
\end{figure*}



\section{The Paraphrase Database}

In this paper, we analyze paraphrases within the Paraphrase Database (PPDB) \cite{ganitkevitch-vandurme-callisonburch:2013:NAACL-HLT}, currently the largest available collection of paraphrases.  Compared to other paraphrase resources such as the DIRT database (12 million rules) \cite{lin2001dirt} and the MSR paraphrase phrase table (13 million) \cite{dolan2004unsupervised}, PPDB contains over 150 million paraphrase rules. These rules were extracted automatically using \newcite{bannard-callisonburch:2005:ACL}'s ``bilingual pivoting'' method, in which two English phrases are assumed to be paraphrases if they both translate to the same foreign phrase.  This is illustrated in Figure \ref{bilingual-pivoting}.  Entries in the PPDB are represented as synchronous context free grammar rules, following \newcite{ganitkevitch-EtAl:2011:EMNLP}.

PPDB has several types of paraphrases: one-to-one (synonyms), one-to-many, many-to-many phrasal paraphrases, and syntactic transformations. We examine the one-to-many paraphrases, since we wanted to analyze whether these paraphrases would be a suitable resource for interpreting multi-word expressions (MWEs).\footnote{Many-to-many paraphrases might be interesting for examining interpretations of other MWEs like noun-noun compounds (e,g. if ``hand lotion'' paraphrased  to ``lotion for hands" or ``apple juice'' paraphrased to ``juice made from apples''). The syntactic paraphrases might also give hints at 
the level of syntactic fixedness of idioms, for instance. 
For this analysis we investigate only the one-to-many paraphrases as a source of potential paraphrases for multiword expressions.}
The PPDB is distributed as a set of six increasingly large files, from Small to XXXL. Each paraphrase in the PPDB is scored according to how precise of a paraphrase it is likely to be. The smaller files contain better-scoring paraphrases, while the larger files contain incrementally more paraphrases, at the cost of precision. For this analysis we chose the Large corpus as a compromise point between coverage and quality of paraphrases.  PPDB-Large contains 188,000 one-to-many paraphrases, and PPDB-XXXL contains 1.5 million of such paraphrases. 


%The Paraphrase Database (PPDB) is a database containing English paraphrases. PPDB was developed using alignment techniques from machine translation on bilingual parallel corpora, pivoting on a foreign language, to find English phrases that translate to the same foreign-language phrase \cite{bannard-callisonburch:2005:ACL}. The database takes into account syntactic information of both the English and foreign-language phrases: the entries in PPDB were found using SCFGs to come up with paraphrases that form constituents of the same syntactic category \cite{ganitkevitch-EtAl:2011:EMNLP,ganitkevitch-vandurme-callisonburch:2013:NAACL-HLT}.

\section{Related Work}

Various definitions of MWEs have been used for NLP tasks. \newcite{Sag2002} define a taxonomy of multiword expressions that is widely used in computational MWE research. They define broad categories for MWEs (fixed expressions, semi-fixed expressions, syntactically flexible expressions, institutionalized phrases), and more specific categories within each of these. The following is a brief outline of their specific categories.   We use an expanded set of categories to analyze the MWEs found in PPDB in section \ref{analysis}.

\begin{itemize}
\item Fixed expressions.  The meaning of fixed expressions is non-compositional, and they do not not follow standard grammar rules.  An example is the phrase ``all of a sudden," which means ``suddenly." The word sudden cannot be replaced with another adjective: ``all of a happy" is nonsensical. Other examples of fixed expressions are ``in short" and ``kingdom come".

\item Non-decomposable idioms. These are non-compositional expressions that do follow grammatical rules. For example, the grammatical structure of ``kick the bucket,'' is valid (e.g., there exist other acceptable English phrases with the same structure, like ``kick the stone,''), however the idiom�s meaning is not sum of its parts. Other examples in- clude ``to take the bull by the horns'' and ``to beat swords into plowshares.'' (Nunberg, Sag, Wasow).***

\item Compound nominals. These are noun combinations whose meaning is not automatically inferable from the individual nouns in the phrase. Contrast ``orange juice'', ``hand lotion'' and ``newspaper column.'' Without word knowledge it is not possible to know whether the nouns� relationship is {\it is made from}, {\it is intended for}, or {\it is located in}.  
%The relationship between the nouns in the compound nominal cannot be deduced from their order in the phrase.

\item Proper nouns. These are nouns that are names for unique entities including places (California, the Bronx), people (President Roosevelt), and events (the Industrial Revolution). This category is included in the taxonomy because names allow for some kinds of variation but not others. For example, {\it the San Francisco 49ers, the 49ers,} and {\it 49ers} are all valid names for the sports team, but ``the Bay Area 49ers" is not.

\item Decomposable idioms. These are phrases whose meaning is compensational from the meaning of the individual words, but which still carry  an idiomatic meaning. One example is ``play with fire."  Decomposable idioms can undergo syntactic changes to varying degrees. For example, the phrase ``let the cat out of the bag" can be modified to ``the cat was out of the bag," and also to ``the cat was \textit{really} out of the bag," adding an adverb. 

\item Verb-particle constructions. These are verb phrases formed of a verb followed by a particle, where the meaning of the phrase is different than that of the verb alone, or up the verb and the particle combined. Some examples are ``wash out," ``break down," and ``follow up." 

\item Light verbs. The combination of light verbs (make, give, take) followed by a normalized verb 
%(a compromise, a presentation, a walk) 
are idiosyncratic in that it is difficult to predict which light verb will be used with the original verb.  ``To walk" can be paraphrased as ``to take a walk," but not ``to make a walk," whereas ``to make a presentation" is fine and ``to take a presentation" nonsensical.
%The verb of the phrase is inflected to change tense (``made a compromise''). 
\end{itemize}

Another piece of related work examined the possibility of using paraphrases to interpret MWEs.  \newcite{dpm} examined the use of paraphrases to better align text-hypothesis pairs in the task of Recognizing Textual Entailment \cite{dagan2006pascal}.  They analyzed paraphrase resources derived from \newcite{lin2001dirt}'s DIRT algorithm and from \newcite{bannard-callisonburch:2005:ACL}'s bilingual pivoting method.  They found that about one third of the MWEs in the RTE data set were contained in the paraphrase resources, but that the MWEs were sometimes mis-paraphrased, making it unclear whether the resources would help with RTE-alignment.  Like \newcite{dpm}, we do an analysis of a paraphrase resource derived from \newcite{bannard-callisonburch:2005:ACL}'s bilingual pivoting method, but at a larger scale.

%Other work has described a division of MWEs extracted from paraphrases: one-to-many, many-to-one,  decomposable and non-decomposable \cite{dpm}. They also investigated using paraphrases to automatically extract multiword expressions. \cite{dpm} consider dependency-based paraphrases and parallel corpora-based paraphrases as sources for MWEs, finding that 34\% of the MWEs in the MSR manual word alignments of the RTE 2006 corpus were present in these resources (Brockett, 2007). 

\section{Analysis of PPDB}\label{analysis}
We analyzed the paraphrases in PPDB to determine the prevalence of MWEs.  We focused on one-to-many paraphrases, since we were interested in whether PPDB could be used to interpret MWEs by replacing them with their single word counterparts. 
%The Paraphrases Database (PPDB) contains English paraphrases. We have characterized a subset of the paraphrases found in the PPDB, according to categories of multi-word expressions (MWEs), syntactic changes in the expansion from a word to its paraphrase, and what parts of speech appear in the corpus. We also looked at how many of the paraphrases in the PPDB appear to be spurious.
We took a random  sample of 500 paraphrases from the PPDB Large one-to-many paraphrase file.  We categorized each of the items in the sample to see how many fell into the MWE categories defined by \newcite{Sag2002}.  For phrases that did not fall within the scope of Sag et al's MWE definitions, we added our own categorizes.  This allowed us to characterize the PPDB's one-to-many.


%The categories of MWEs we looked at were light verbs, verb-particle constructions, negation, and superlatives. We also included the categories for MWEs described in Sag et al: fixed expressions, non-decomposable idioms, compound nominals, proper names, and decomposable idioms. 

%While categorizing the paraphrases we found that paraphrases sometimes did not fall under the classic MWE categories, but they were valid paraphrases because of a change in syntax between the two phrases, or because of some other known relationship between paraphrases (e.g., acronyms). Therefore, in addition to the MWE categories, 
In addition to Sag et al's inventories, we added 21 additional categories.  6 of our categories were meaning-preserving transformations that did not change the syntax of the phrases. 

\begin{itemize}

\item Acronyms. This category captures paraphrases where one or both of the entries in a paraphrase pair are acronyms. Usually the acronym is expanded (e.g., ``sme/small and medium enterprises" or ``unicef/united nations children 's fund'').  Sometimes the expansion contains a variant acronym (e.g., ``cmf," ``the fcm'').
%sme | small and medium enterprises
%unicef | united nations children 's fund
%roi | return on investment

\item Negation.  The phrase component of these included, ``not'' followed by another word.  Their single word counterpart contained a negative prefix.   For instance ``unacceptable/not acceptable'', ``non-toxic/not toxic'', and ``illegible/not readable''.

\item Superlatives and comparatives. The phrase components of these included ``most'' or ``more'' followed by an adjective.  The one word paraphrase contained the suffix ``-est" or ``-er''.  For example, ``most recent/latest'', ``most high-ranking/highest'', ``more humid/wetter'', ``more sexy/sexier''.
%
%Superlatives @ppdb:
%latest | most recent
%cruellest | most cruel
%oldest | most ancient
%bravest | most courageous
%highest | most high-ranking
%
%safer | more secure
%denser | more dense
%deader | more dead
%wetter | more humid
%fairer | more just
%sexier | more sexy


%\item Hypernym-hyponym pairs. This category was included to capture paraphrases where either the original word was a hypernym of the paraphrase, or vice versa. There were no instances in the random sample.

\item Time variation. Paraphrases in this category were contained either times (e.g., ``9:00," ``9 a.m.'') or dates (``2003/04," ``the 2003-04 fiscal year'') in either the original word or the paraphrase.  Some of these could be rewritten in many different ways.  For instance, ``9:00'' is rewritten not only as ``9 a.m.'' but also as ``9 hours'', ``9.00 a.m.'', `` 9:00 am'', ``nine hours'' and ``nine o 'clock''.
%09:00 | 9 a.m.
%09:00 | 9 hours
%09:00 | 9.00 a.m.
%09:00 | 9:00 am
%09:00 | nine hours
%09:00 | nine o 'clock

\item  Variant use of punctuation. This category captures paraphrases where either the original word or the expansion contains a punctuation mark (e.g., apostrophe or hyphen), which is what causes the word to be interpreted as a single word (e.g., ``debt-servicing," ``debt servicing'') or the expansion to be interpreted as having multiple words (e.g., ``what," ``somethin ' '').

\item Numbers. Paraphrases in this category contain number in either the original word or its expansion. Some examples are ``20," ``twenty of," and ``5,000," ``five thousand."

\end{itemize}


We added 6 categories for syntactic changes between the word and its paraphrase.  Many of these result from the fact that the syntactic constraints that are imposed on the paraphrases are not fine grained \cite{ccb:2008:EMNLP}.  Each phrase and its paraphrase must have the same syntactic categories.  PPDB's syntactic categories are derived from the Penn Treebank tag set, so all verb phrases are grouped together into the label VP, regardless of their tense or whether they include modals. 
\begin{itemize}
\item Change of tense. Paraphrases in this category both rephrased the original word and changed the tense of the phrase. An example is the pair ``say", ``going to tell". 

\item Nominalization. Paraphrases in this category had one of the phrases as a noun form of a verb. Two examples are ``a monopoly" and ``monopolization", and ``operation" and  ``proper functioning." 

\item Infinitival to. Paraphrases in this category had the infinitival form a verb as the expansion, and either the same verb or one with similar meaning as the one-word paraphrase. One example is ``answer," ``to reply."

\item Adverbial modifier. Paraphrases in this category had either the original word or a word with similar meaning, plus an adverbial modifier. Two examples are ``interesting", ``very interesting," and ``teeny", ``really little."

\item Partial word overlap. 
%This category captures paraphrases where some of the paraphrase is the same as some of the original word, but it is not simply the original word with additional modifiers. 
This category is meant to capture paraphrases that are lexicographically related to each other even if the original word is not a substring of the paraphrase. Some examples of this category are ``anytime", ``any point," and ``vice-president," ``the deputy president."

\item Added determiner. Paraphrases in this category map one word to a phrase consisting of a determiner followed by a one-word paraphrase of the original word. Some examples are ``photographs," ``the images," and ``mist," ``the smoke."  This also contains generics like ``loans," ``a loan."
%
%\item Change of tense. Paraphrases in this category have the same verb in both the original word and its expansion, but the tense is changed. Some examples are ``changing," ``be changed," and ``attain", ``be attained."
\end{itemize}

Finally, we defined 2 categories to capture paraphrases that did not fall under the classic MWE, syntactic change, or non-syntactic categories.
\begin{itemize}
\item Bad paraphrases. These are paraphrases where the original word and the expansion do not seem to be related in any way, or where there is no context where the expansion could be replaced by the original word. 
\item Unspecified expansions. These paraphrases do not fall under any of the other paraphrase categories defined until now. 
\end{itemize}


\begin{center}
\begin{table*}
\hfill{}
\begin{tabular}{|l|l|p{.5\linewidth}|}
\hline
\bf Paraphrase Category & \bf Instances & \bf Examples \\ 
 & \bf in sample & \\
\hline
%expansion & 62 & suffocate, need air; cumbersome, time consuming \\
%expansion (same morphological form) & 47 & fun, sounds like fun; signs, signs and signals \\
paraphrastic expansions & 109 & suffocate, need air; cumbersome, time consuming, fun, sounds like fun; signs, signs and signals \\
added determiner & 95 & photographs, the images; duties, the responsibilities; militaries, the military \\
%added determiner & 86 & photographs, the images; duties, the responsibilities \\
%determiner + plural & 9 & gloves, the glove; militaries, the military\\
change of tense & 50 & initiated, has undertaken; say, going to tell; changing, be changed\\
%change of tense + paraphrase & 43 & initiated, has undertaken; say, going to tell \\
%change of tense & 7 & changing, be changed; attain, be attained\\
inaccurate/bad paraphrase & 43 & iron, a par; also, do i\\
implicit/most common modifier & 41 & baghdad, iraqi capital baghdad; enrichment, uranium enrichment \\
implicit type & 29 & training, training course; customs, customs offices \\
adverbial modifier& 27 & interesting, very interesting; more, even more \\
acronym & 21 & gpa, the global programme of action; cras, \\
&& credit-rating agencies \\
partial word overlap  & 17 &  anytime, any point; enslavement, slave labour \\
\bf verb-particle & 14  & torched, burnt down; done, carried out\\
\bf proper noun & 7 & markov, mr markov; karadzic, radovan karadzic\\
\bf fixed expression & 6 & applied, put into effect; plenty, a whole host \\
superlative & 5 & notably, most particularly; best-known, most famous\\
copula & 5 & qualify, are eligible; reason, been right \\
\bf decomposable idiom & 4 & nuts, out of your mind; sleeping, get a good night's sleep\\
\bf light verb & 4  &  issued, made available; place, make way \\
number & 4 & 20, twenty of; 5,000, 5 000\\
infinitival to & 3 & track, to follow; answer, to reply \\
nominalization & 3 &  operation, proper functioning \\
negation & 2 & unused, not utilized; non-parties, not parties  \\
time variation & 2 & 7:00, seven hours; 2003/04, the 2003-04 fiscal year\\
punctuation & 2 & debt-servicing, debt servicing; what, somethin '\\
\bf non-decomposable idiom & 2 & furious, as mad as hell; entails, brings with it\\
alternate spelling & 1 & al-najaf, al nagaf\\
\bf compound nominal & 0 &\\
\hline
\end{tabular}
\hfill{}
\caption{\label{big-table} We manually categorized a random sample of 500 one-to-many paraphrases drawn from PPDB.  The categories in bold are drawn from \newcite{Sag2002}'s taxonomy of MWEs.  }
\end{table*}
\end{center}



\begin{center}
\begin{table*}
\hfill{}
\begin{tabular}{|l|l|l|p{.5\linewidth}|}
\hline \bf Light verb & \bf Manually  & \bf Number  & \bf Example matches, and their paraphrases\\ 
 & \bf verb phrases & \bf  categorized & \\ 
 & \bf  found by regex & \bf as light verbs & \\ \hline
make & 46 & 35 & make a call, make a decision, make a suggestion\\
have & 105 & 45 & have a ball, have a conversation, have a word\\
give & 11 & 4 & give a damn, give a reply\\
take & 48 & 33 & take a break, take a look, take a walk\\
hold & 3 & 1 & hold a debate\\
\hline
\end{tabular}
\hfill{}
\caption{In addition to taking a random sample of the multi-word paraphrases in PPDB, we directly searched for classic MWEs using regular expressions. This table shows examples of the light verbs that matched our regex. }
\label{light-verbs}
\end{table*}
\end{center}

Table \ref{big-table} gives the results of our manual categorization of our 500 randomly selected examples.  
The two most common types of paraphrases were paraphrastic expansions of single words into multiword phrases (109 instances, or 22\%) 
 the addition of a determiner (95 instances or 19\%). Of the sample, 43 were bad paraphrases (8.6\%).  Of the 500 paraphrases in the random sample, only 37 (7.4\%) fell into the classic MWE categories defined by \newcite{Sag2002}. Of these MWE paraphrases, the most common were verb-particle constructions (14 instances, or 37.8\%) , followed by proper names (7, or 18.9\%). There were no instances of compound nominals in the sample.




% CCB  - I am not sure that this adds anything. 
%
%The distribution of the syntactic categories from this random sample is depicted in Table \ref{sc-dist}.  
%In both samples, the most common part of speech is NP, followed by VP. 
%
%\begin{table}
%\hfill{}
%\begin{center}
%\begin{tabular}{|l|l|l|}
%\hline \bf Syntactic  & \bf Freq in  & \bf Freq in  \\ 
%\bf  category & \bf sample & \bf PPDB-L \\ \hline
%NP & 254 & 101563 \\
%VP & 109 & 39380 \\
%X & 39 & 17005 \\
%ADJP & 27 & 14513 \\
%ADVP & 16 & 5451 \\
%S & 9 & 4103 \\
%INTJ & 3 & 1730 \\
%TOP & 4 & 1476 \\
%NX & 2 & 1113 \\
%PP & 1 & 532 \\
%FRAG & 3 & 388  \\ 
%SQ & 0 & 305 \\
%SBAR & 0 & 165 \\
%WHNP & 0 & 163 \\
%WHADVP & 0 & 66 \\
%QP & 0 & 39 \\
%LST & 0 & 3 \\ 
%SBARQ & 0 & 3 \\
%CONJP & 0 & 1 \\
%PRT & 0 & 1 \\
%\hline
%\end{tabular}
%\end{center}
%\caption{\label{sc-dist} Distribution of syntactic categories in random sample of 500 words from PPDB L, and in the entire PPDB L file, sorted by frequency in PPDB L. }
%\end{table}
%


\subsection{Directly locating MWEs}

In addition to categorizing a random sample of paraphrases, we searched for instances of light verbs and verb-particle constructions. The light verbs were those with the verb ``have'', ``take'', ``make'', ``hold'' or ``give'' followed by a noun phrase. The verb-particle constructions were any verbs followed by the particles ``down'', ``up'', ``on'', ``out'', ``over'' or ``upon''. 
%Negation instances had the word ``not'' either in the original or the expanded paraphrase. Comparatives had the word ``more,'' and superlatives had the word ``most.''

To find instances of verb-particle constructions, we constructed a regular expression to match a list of particles at the end of a phrase. The regex {\tt particle\textbackslash b} matched phrases where the potential particle was not the first word, and ensured that it was not a substring of another word (e.g., `onto' instead of `on'). For example, to find potential instances of verb-particle constructions with the particle `up', we searched for the following regular expression: {\tt up\textbackslash b}.

We then manually determined  whether the phrases found were verb-particle constructions. The potential particle was sometimes a preposition or adverb, in which case it did not fit into this MWE category. 
The results from these searches are summarized in the Tables \ref{light-verbs} and \ref{verb-particle}. We considered only light verb phrases with no extra terms (e.g., `have a word', but not `have a word with you' or `can i have a word').


\begin{table*}
\begin{center}
\begin{tabular}{|l|l|l|p{.5\linewidth}|}
\hline \bf Verb particle & \bf Found  & \bf Number  & \bf Example matches, and their paraphrases\\ 
 & \bf   by regex & \bf correct & \\ \hline
up & 439 & 439 & \\
about & 115 & 10 & \\
around & 47 & 44 \\
back & 156 & 145 \\
down & 195 & 190 \\
in & 207 & 32 \\
off & 134 & 134 \\
on & 196 & 109 \\
out & 383 & 383 \\
over & 67 & 65 \\
\hline
\end{tabular}
\end{center}
\caption{\label{verb-particle} Verb-particle constructions found in PPDB by our regular expression search. }
\end{table*}




%
%We used similar commands to find instances of the other paraphrase categories. A list of the commands used is in Table \ref{grep}:
%
%\begin{center}
%\begin{table*}[h]
%\hfill{}
%\begin{tabular}{|l|rl|}
%\hline \bf Paraphrase category & \bf Search command & \\ \hline
%verb-particle construction & grep ' up \textbackslash b' ppdb\textunderscore filename &\\
%light verbs & grep '\textbackslash b make ' ppdb\textunderscore filename &\\
%negation & grep '\textbackslash b not ' ppdb\textunderscore filename &\\
%comparatives & grep '\textbackslash b more ' ppdb\textunderscore filename &\\
%superlatives & grep '\textbackslash b most ' ppdb\textunderscore filename &\\
%\hline
%\end{tabular}
%\hfill{}
%\caption{\label{grep} Linux commands to search for paraphrases of different types. }
%\end{table*}
%\end{center}



\section{Building a classifier for MWEs in PPDB}


Among the 500 paraphrases in the sample, 37 fell into the category of interesting MWEs. Since the PPDB L one-to-many corpus contains 188,000 rules, based on our sample, we expect there to be around 14,000 relevant MWEs in the database. 188,000 rules is too many to manually sift through, and the PPDB XXXL corpus has far more rules (1.5 million). Therefore, we built a classifier to automatically distinguish paraphrases that are interesting to MWE researchers.

\subsection{Experimental setup}

We designed features tailored to find paraphrases in interesting categories of MWE. The methodology used in the classifier could also be applied more generally for finding relevant expressions in PPDB (for example, trying to automatically classify hypernym/hyponym pairs instead of MWEs).

The features we used were:
\begin{itemize}
\item A binary feature that indicated whether a preposition appeared in the expansion of the paraphrase (for prepositions  `about', `around', `back', `down', `in', `off', `on', `out', `over', `up' )
\item A binary feature to indicate whether the original word appeared in the expansion
\item A binary feature to indicate whether a light verb appeared in the expansion (for  `give', `have', `hold', `make', `take'), 
\item All of the scores from the PPDB.  There are 31 scores associated with each entry in the PPDB.  These features help to score the goodness of paraphrases and are described in \newcite{Ganitkevitch-Callison-Burch-2014:LREC}.
\end{itemize}

We used the 500 manually classified examples from Table \ref{big-table} as training and test data, performing a ten-fold cross-validation.  We performed a binary classification to distinguish between ``interesting" MWEs and the other paraphrases in the set.  We denoted all of the \newcite{Sag2002} MWEs (verb-particle construction; expansion, same morphological form; light verb; fixed expression; non-decomposable idiom; proper noun; decomposable idiom; light verb) as interesting plus two others implicit/most common modifier and implicit type.  The majority class was paraphrases that were not interesting to MWE researchers.  This consisted of 69.2\% of the data, and the interesting MWEs consisted of 30.8\%.

\subsection{Results}

We used an SVM classifier (libsvm) with these features on our sample of 500 words from the PPDB L corpus. Averaged over the ten-fold cross-validation, the accuracy of the classifier was 76.4\%. This improves over the majority class baseline by 7.2\% absolute / 10.4\% relative.

\section{Using paraphrased MWEs for an NLP task}

\subsection{Experimental design}

To determine whether a lexicon of paraphrases extracted from the PPDB would be useful for the task of parsing, we took all of the 33 paraphrases from our sample that fell into the categories described by \newcite{Sag2002}, as well as one paraphrase from each of the other categories from the random sample of 500 words. We then did a Google search to find a sentence containing the expansion, selecting sentences where the part of speech matched that of the paraphrase, by manual inspection. We restricted the sentences to those less than 50 words long. We then created an equivalent sentence by replacing the expansion with the one-word paraphrase. We used the online demo of the Berkeley parser \cite{petrov-EtAl:2006:COLACL} to parse these sentence pairs, and manually evaluated the resulting parse trees for each sentence of the pair.

\subsection{Results}

Of the 33 paraphrases sampled from MWE categories, 30 were correctly parsed using the one-word paraphrase, and 29 were correctly parsed using the expansion. 2 were parsed correctly for the one-word paraphrase but not for the expansion, 1 was parsed correctly for the expansion but not the one-word paraphrase, and 1 was parsed incorrectly for both. Figure \ref{avail} shows the trees for a case where the sentence containing the one-word paraphrase (``leverage'') is parsed correctly, but the sentence containing the expansion (``avail myself of'') is not. Figure \ref{issued} shows the trees for a case where the sentence containing the one-word paraphrase (``issued'') is parsed correctly, but the sentence containing the expansion (``made available'') is not. Figure \ref{place} shows the trees for a case where both sentences are parsed incorrectly.

Of the 18 other paraphrases sampled, all were parsed correctly in their original form, while 15 were parsed correctly using the one-word paraphrase.  This suggests that it is important to carefully choosing which paraphrases to use and when they should be substituted.  Note that this negative result may be attributable to our method for randomly sampling paraphrases.  The PPDB contains many paraphrases for the same input phrase, and these can be sorted from best to worst.  We randomly draw any of the paraphrases from each original phrase, suggesting that a better paraphrase may exist within the set.

\begin{figure*}
\centering
\begin{subfigure}{.5\textwidth}
  \centering
  \includegraphics[width=80mm]{figs/avail_myself_of_tree.png}
  \caption{Original sentence: I'll avail myself of the first cab \\ to come along.}
  \label{fig:sub1}
\end{subfigure}%
\begin{subfigure}{.5\textwidth}
  \centering
  \includegraphics[width=58mm]{figs/leverage_tree.png}
  \caption{Sentence with substitution: I'll leverage the first cab to come along.}
  \label{fig:sub2}
\end{subfigure}
\caption{Parse trees for a sentence pair where substituting the single-word paraphrase for the MWE (in figure (b)) yields a better parse than the original sentence (in figure (a)).}
\label{avail}
\end{figure*}


\begin{figure*}
\centering
\begin{subfigure}{.5\textwidth}
  \centering
  \includegraphics[width=80mm]{figs/made_available_tree.png}
  \caption{Original sentence: A new vaccine against the deadly \\
disease meningitis is set to be made available free \\
on the NHS.}
  \label{fig:sub1}
\end{subfigure}%
\begin{subfigure}{.5\textwidth}
  \centering
  \includegraphics[width=73mm]{figs/issued_tree.png}
  \caption{Sentence with substitution: A new vaccine against the deadly disease meningitis is set to be issued free on the NHS.}
  \label{fig:sub2}
\end{subfigure}
\caption{Parse trees for a sentence pair where the original sentence (in figure (a)) yields a better parse than substituting the single-word paraphrase for the MWE (in figure (b)).}
\label{issued}
\end{figure*}


\begin{figure*}
\centering
\begin{subfigure}{.5\textwidth}
  \centering
  \includegraphics[width=80mm]{figs/make_way_tree.png}
  \caption{Original sentence:See how workers are gutting the \\
 580 Building to make way for luxury apartments.}
  \label{fig:sub1}
\end{subfigure}%
\begin{subfigure}{.5\textwidth}
  \centering
  \includegraphics[width=87mm]{figs/place_tree.png}
  \caption{Sentence with substitution: See how workers are gutting the 580 Building to place for luxury apartments.}
  \label{fig:sub2}
\end{subfigure}
\caption{Parse trees for a sentence pair where the original sentence (in figure (a)) yields a better parse than substituting the single-word paraphrase for the MWE (in figure (a)).}
\label{place}
\end{figure*}

\section{Summary}
We investigated whether the PPDB could be a useful source for multiword expressions. To achieve this goal, we looked at a random sample of 500 paraphrases from the database and categorized them according to widely recognized categorized of multiword expressions, as well as other categories describing the relationship between the two phrases in a paraphrase. We then developed a classifier to automatically extract MWEs from PPDB that MWE researchers might find relevant, so that relevant MWEs could be found easily from the entire PPDB file, which would be too large to scan manually. Finally, we did a pilot study on the paraphrases in PPDB, replacing a multiword phrase in a sentence with its one-word paraphrase from PPDB, and then parsing both sentences to see if parsing would improve after paraphrasing. We found that there was no significant difference between the parses for either the expansions or the one-word paraphrases.

%
% We are not doing the following for Coling 2014. Maybe next time.
%
% \section{XML conversion and supported \LaTeX\ packages}
% 
% ACL 2014 innovates over earlier years in that we will attempt to
% automatically convert your \LaTeX\ source files to machine-readable
% XML with semantic markup. This will facilitate future research that
% uses the ACL proceedings themselves as a corpus.
% 
% We encourage you to submit a ZIP file of your \LaTeX\ sources along
% with the camera-ready version of your paper. We will then convert them
% to XML automatically, using the LaTeXML tool
% (\url{http://dlmf.nist.gov/LaTeXML}). LaTeXML has \emph{bindings} for
% a number of \LaTeX\ packages, including the ACL 2014 stylefile. These
% bindings allow LaTeXML to render the commands from these packages
% correctly in XML. For best results, we encourage you to use the
% packages that are officially supported by LaTeXML, listed at
% \url{http://dlmf.nist.gov/LaTeXML/manual/included.bindings}

%\section*{Acknowledgements} 
%This material is based on research sponsored by the NSF under grant IIS-1249516 and DARPA under agreement number FA8750-13-2-0017 (the DEFT program). The U.S. Government is authorized to reproduce and distribute reprints for Governmental purposes. The views and conclusions contained in this publication are those of the authors and should not be interpreted as representing official policies or endorsements of DARPA or the U.S. Government.

% include your own bib file like this:
\section*{Bibliography}
\bibliographystyle{acl}
\bibliography{acl2014}

\end{document}
