%
% File coling2014.tex
%
% Contact: jwagner@computing.dcu.ie
%%
%% Based on the style files for ACL-2014, which were, in turn,
%% Based on the style files for ACL-2013, which were, in turn,
%% Based on the style files for ACL-2012, which were, in turn,
%% based on the style files for ACL-2011, which were, in turn, 
%% based on the style files for ACL-2010, which were, in turn, 
%% based on the style files for ACL-IJCNLP-2009, which were, in turn,
%% based on the style files for EACL-2009 and IJCNLP-2008...

%% Based on the style files for EACL 2006 by 
%%e.agirre@ehu.es or Sergi.Balari@uab.es
%% and that of ACL 08 by Joakim Nivre and Noah Smith

\documentclass[11pt]{article}
\usepackage{coling2014}
\usepackage{times}
\usepackage{url}
\usepackage{latexsym}
\usepackage{graphicx}

%\setlength\titlebox{5cm}

% You can expand the titlebox if you need extra space
% to show all the authors. Please do not make the titlebox
% smaller than 5cm (the original size); we will check this
% in the camera-ready version and ask you to change it back.


\title{Instructions for COLING-2014 Proceedings}

\author{First Author \\
  Affiliation / Address line 1 \\
  Affiliation / Address line 2 \\
  Affiliation / Address line 3 \\
  {\tt email@domain} \\\And
  Second Author \\
  Affiliation / Address line 1 \\
  Affiliation / Address line 2 \\
  Affiliation / Address line 3 \\
  {\tt email@domain} \\}

\date{}

\begin{document}
\maketitle
\begin{abstract}
In this paper we present a feasibility study for rewriting multiword expressions as single words, which NLP systems could potentially process more easily than the original phrases. Here we investigate PPDB: The Paraphrase Database to get a mapping from multiword expressions onto single words, using the MWE categorization system as described in Baldwin, et al. 
\end{abstract}

\section{Introduction}
\label{intro}

%
% The following footnote without marker is needed for the camera-ready
% version of the paper.
% Comment out the instructions (first text) and uncomment the 8 lines
% under "final paper" for your variant of English.
% 
\blfootnote{
    %
    % for review submission
    %
    \hspace{-0.65cm}  % space normally used by the marker
    Place licence statement here for the camera-ready version, see
    Section~\ref{licence} of the instructions for preparing a
    manuscript.
    %
    % % final paper: en-uk version (to license, a licence)
    %
    % \hspace{-0.65cm}  % space normally used by the marker
    % This work is licensed under a Creative Commons 
    % Attribution 4.0 International Licence.
    % Page numbers and proceedings footer are added by
    % the organisers.
    % Licence details:
    % \url{http://creativecommons.org/licenses/by/4.0/}
    % 
    % % final paper: en-us version (to licence, a license)
    %
    % \hspace{-0.65cm}  % space normally used by the marker
    % This work is licenced under a Creative Commons 
    % Attribution 4.0 International License.
    % Page numbers and proceedings footer are added by
    % the organizers.
    % License details:
    % \url{http://creativecommons.org/licenses/by/4.0/}
}

Multiword expressions (MWEs) are phrases whose meanings are different than the literal interpretation of the words in the phrase. MWEs include verb-particle constructions, fixed expressions, compound nominals, and decomposable idioms, to name a few. 

MWEs are difficult for non-native speakers of English to understand, and also for NLP systems to identify. 


\section{Experimental Design}
The Penn Paraphrases Database (PPDB) contains English paraphrases. We have characterized a subset of the paraphrases found in the PPDB, according to categories of multi-word expressions (MWEs), syntactic changes in the expansion from a word to its paraphrase, and what parts of speech appear in the corpus. We also looked at how many of the paraphrases in the PPDB appear to be spurious.

The categories of MWEs we looked at were light verbs, verb-particle constructions, negation, and superlatives. We also included Tim Baldwin's categories for MWEs: fixed expressions, non-decomposable idioms, compound nominals, proper names, and decomposable idioms. 

In addition to MWE categories, we also included categories for syntactic changes from a word to its paraphrase: change of tense followed by a paraphrase, nominalizations, infinitival to, adverbial modifier, one or more words the same as part of the original word, determiner followed by a one-word paraphrase, determiner followed by the plural form, and change of tense. Finally, we included acronyms, hypernym-hyponym pairs, times, extra punctuation marks, and numbers as categories, as well as unspecified expansions and bad paraphrases.

\section{Results}

Of a random sample of 500 paraphrases from the L one-to-many paraphrase file, the most common types of paraphrase were expansions using the same morphological form (117 instances, or 23.4%), determiner followed by a one-word paraphrase (86 instances, or 17.2%), and paraphrases that did not fall into a particular category (62 instances, or 12.4%). Of the sample, 43 were bad paraphrases (8.6%). The full list of categories and the number of instances in each are in the table below.  

The distribution of the parts of speech from this random sample is depicted in the histogram below. 

\begin{center}
\includegraphics[width=150mm]{figs/random_sample_pos_histogram_500.png}
\end{center}

The distribution of all of the parts of the speech from the L one-to-many paraphrase file is depicted in the following histogram:

\begin{center}
\includegraphics[width=175mm]{figs/random_sample_pos_histogram_all.png}
\end{center}

In both samples, the most common part of speech is NP, followed by VP. 

Below are illustrative examples of all categories of multiword expressions:

\begin{table}[h]
\begin{center}
\begin{tabular}{|l|rl|}
\hline \bf MWE Category & \bf Example & \\ \hline
verb-particle & torched, burnt down  &\\
fixed expression & applied, put into effect &\\
non-decomposable idiom & furious, as mad as hell &\\
proper noun & markov, mr markov &\\
decomposable idiom & nuts, out of your mind &\\
light verb & issued, made available  &\\
superlative & notably, most particularly &\\
negation & unused, not utilized &\\
\hline
\end{tabular}
\end{center}
\caption{\label{font-table} Examples from each category. }
\end{table}

In addition to categorizing a random sample of paraphrases, I searched for instances of light verbs, verb-particle constructions, negation, comparatives, and superlatives. The light verbs were those with the verb “have”, “take”, “make”, “hold” or “give, ” followed by a noun phrase. The verb-particle constructions were any verbs followed by the particles “down”, “up”, “on”, “out”, “over” or “upon.” Negation instances had the word “not” either in the original or the expanded paraphrase. Comparatives had the word “more,” and superlatives had the word “most.”

The results from these searches are summarized in the table below.



%
% We are not doing the following for Coling 2014. Maybe next time.
%
% \section{XML conversion and supported \LaTeX\ packages}
% 
% ACL 2014 innovates over earlier years in that we will attempt to
% automatically convert your \LaTeX\ source files to machine-readable
% XML with semantic markup. This will facilitate future research that
% uses the ACL proceedings themselves as a corpus.
% 
% We encourage you to submit a ZIP file of your \LaTeX\ sources along
% with the camera-ready version of your paper. We will then convert them
% to XML automatically, using the LaTeXML tool
% (\url{http://dlmf.nist.gov/LaTeXML}). LaTeXML has \emph{bindings} for
% a number of \LaTeX\ packages, including the ACL 2014 stylefile. These
% bindings allow LaTeXML to render the commands from these packages
% correctly in XML. For best results, we encourage you to use the
% packages that are officially supported by LaTeXML, listed at
% \url{http://dlmf.nist.gov/LaTeXML/manual/included.bindings}


\section{Analysis}

\section*{Acknowledgements}

The acknowledgements should go immediately before the references.  Do
not number the acknowledgements section. Do not include this section
when submitting your paper for review.

% include your own bib file like this:
%\bibliographystyle{acl}
%\bibliography{acl2014}

\begin{thebibliography}{}

\bibitem[\protect\citename{Aho and Ullman}1972]{Aho:72}
Alfred~V. Aho and Jeffrey~D. Ullman.
\newblock 1972.
\newblock {\em The Theory of Parsing, Translation and Compiling}, volume~1.
\newblock Prentice-{Hall}, Englewood Cliffs, NJ.

\bibitem[\protect\citename{{American Psychological Association}}1983]{APA:83}
{American Psychological Association}.
\newblock 1983.
\newblock {\em Publications Manual}.
\newblock American Psychological Association, Washington, DC.

\bibitem[\protect\citename{{Association for Computing Machinery}}1983]{ACM:83}
{Association for Computing Machinery}.
\newblock 1983.
\newblock {\em Computing Reviews}, 24(11):503--512.

\bibitem[\protect\citename{Chandra \bgroup et al.\egroup }1981]{Chandra:81}
Ashok~K. Chandra, Dexter~C. Kozen, and Larry~J. Stockmeyer.
\newblock 1981.
\newblock Alternation.
\newblock {\em Journal of the Association for Computing Machinery},
  28(1):114--133.

\bibitem[\protect\citename{Gusfield}1997]{Gusfield:97}
Dan Gusfield.
\newblock 1997.
\newblock {\em Algorithms on Strings, Trees and Sequences}.
\newblock Cambridge University Press, Cambridge, UK.

\end{thebibliography}

\end{document}
